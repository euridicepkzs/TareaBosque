\documentclass[12pt a4paper]{article}
\usepackage[utf8]{inputenc}
\usepackage{lipsum}
\title{El Bosque Oscuro}
\author{Euridice}
\date{1 de abril de 2024}
\begin{document}
\maketitle
\section{Resumen}
El video cubre la hipótesis del Bosque Oscuro como una explicación alternativa a la paradoja de Fermi sobre la aparente falta de señales de vida inteligente extraterrestre en el universo. La hipótesis sugiere que las civilizaciones inteligentes evitan el contacto por temor a ser aniquiladas en una lucha por recursos limitados, permaneciendo en silencio y ocultas como cazadores en un bosque oscuro. También se discuten otras posibles explicaciones a la paradoja de Fermi, como que estamos solos en el universo o que los extraterrestres están evitando el contacto intencionalmente. El video analiza los esfuerzos de la humanidad por enviar mensajes a posibles civilizaciones extraterrestres (METI) y cuestiona si esto es una buena idea según la hipótesis del Bosque Oscuro.
\section{La paradoja de Fermi y posibles explicaciones}
Se explica la paradoja de Fermi y se presentan varias posibles explicaciones, como la idea de que estamos solos en el universo o que los extraterrestres evitan el contacto intencionalmente.
\section{La novela "El Bosque Oscuro" de Liu Cixin}
Se menciona la novela escrita por Liu Cixin que inspira la hipótesis del Bosque Oscuro.
\section{El universo como un bosque oscuro}
Se describe la metáfora del universo como un bosque oscuro donde las civilizaciones inteligentes se ocultan como cazadores sigilosos.
\section{La hipótesis del Bosque Oscuro}
Se explica la hipótesis como una explicación alternativa a la paradoja de Fermi sobre la falta de señales de vida extraterrestre.
\subsection{La lucha por recursos limitados}
Si sabemos que los recursos en el universo son limitados entonces podemos plantear dos posibilidades:
\subsubsection{Estamos solos en el universo}
Si estamos solo en el universo no habría una preocupación genuina respecto a que estos recursos en el universo se acaben, solo estariamos frente a una inminente extinción y quedar como lo mejor de la creacion del universo
\subsubsection{Existen muchas otras civilizaciones}
Frente a este panorama podría existir una licha por los recursos naturales donde podriamos observar que pueden existir dos tipos de civilizacion.
\subsection{¿Estamos solos en el universo o los extraterrestres evitan el contacto intencionalmente?}
De acuerdo a la hipotesis del bosque oscuro podriamos decir que evitar contactar con ptras civilizaciones seria lo mas sensato de acuerdo a nuestro instinto de supervivencia.
\subsection{Mensajes a posibles civilizaciones extraterrestres}
A lo largo de los años hubo muchos intentos de mandar señales a estrellas o planteas con el objetivo de saber si existen otros seres en el universo.
\subsubsection{¿Es bueno que se envíen estos mensajes?}
\subsubsection{¿Obtendremos algún día una respuesta?}
\end{document}